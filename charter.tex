\documentclass[
11pt, % The default document font size, options: 10pt, 11pt, 12pt
%codirector, % Uncomment to add a codirector to the title page
]{charter} 


% El títulos de la memoria, se usa en la carátula y se puede usar el cualquier lugar del documento con el comando \ttitle
\titulo{Sistema de monitoreo de calidad del aire} 

% Nombre del posgrado, se usa en la carátula y se puede usar el cualquier lugar del documento con el comando \degreename
%\posgrado{Carrera de Especialización en Sistemas Embebidos} 
\posgrado{Carrera de Especialización en Internet de las Cosas} 
%\posgrado{Carrera de Especialización en Inteligencia Artificial}
%\posgrado{Maestría en Sistemas Embebidos} 
%\posgrado{Maestría en Internet de las cosas}

% Tu nombre, se puede usar el cualquier lugar del documento con el comando \authorname
% IMPORTANTE: no omitir titulaciones ni tildación en los nombres, también se recomienda escribir los nombres completos (tal cual los tienen en su documento)
\autor{Ing. Rodrigo Jurgen Pinedo Nava}

% El nombre del director y co-director, se puede usar el cualquier lugar del documento con el comando \supname y \cosupname y \pertesupname y \pertecosupname
\director{Por definir}
\pertenenciaDirector{pertenencia} 
\codirector{} % para que aparezca en la portada se debe descomentar la opción codirector en los parámetros de documentclass
\pertenenciaCoDirector{FIUBA}

% Nombre del cliente, quien va a aprobar los resultados del proyecto, se puede usar con el comando \clientename y \empclientename
\cliente{Ing. Jose Mauricio Vargas Nuñez}
\empresaCliente{Emprendimiento personal}
 
\fechaINICIO{10 de marzo de 2025}		%Fecha de inicio de la cursada de GdP \fechaInicioName
\fechaFINALPlan{29 de abril de 2025} 	%Fecha de final de cursada de GdP
\fechaFINALTrabajo{30 de junio de 2025}	%Fecha de defensa pública del trabajo final


\begin{document}

\maketitle
\thispagestyle{empty}
\pagebreak


\thispagestyle{empty}
{\setlength{\parskip}{0pt}
\tableofcontents{}
}
\pagebreak


\section*{Registros de cambios}
\label{sec:registro}


\begin{table}[ht]
\label{tab:registro}
\centering
\begin{tabularx}{\linewidth}{@{}|c|X|c|@{}}
\hline
\rowcolor[HTML]{C0C0C0} 
Revisión & \multicolumn{1}{c|}{\cellcolor[HTML]{C0C0C0}Detalles de los cambios realizados} & Fecha      \\ \hline
0      & Creación del documento                                 &\fechaInicioName \\ \hline
1      & Se completa hasta el punto 5 inclusive                 & 20 de marzo de 2025 \\ \hline
2      & Se completa hasta el punto 9 inclusive                 & 28 de marzo de 2025 \\ \hline
%1      & Se completa hasta el punto 5 inclusive                & {día} de {mes} de 202X \\ \hline
%2      & Se completa hasta el punto 9 inclusive
%		  Se puede agregar algo más \newline
%		  En distintas líneas \newline
%		  Así                                                    & {día} de {mes} de 202X \\ \hline
%3      & Se completa hasta el punto 12 inclusive                & {día} de {mes} de 202X \\ \hline
%4      & Se completa el plan	                                 & {día} de {mes} de 202X \\ \hline

% Si hay más correcciones pasada la versión 4 también se deben especificar acá

\end{tabularx}
\end{table}

\pagebreak



\section*{Acta de constitución del proyecto}
\label{sec:acta}

\begin{flushright}
Buenos Aires, \fechaInicioName
\end{flushright}

\vspace{2cm}

Por medio de la presente se acuerda con el \authorname\hspace{1px} que su Trabajo Final de la \degreename\hspace{1px} se titulará ``\ttitle'' y consistirá en {la implementación de una red de sensores especializados para la medición de partículas en suspensión, dióxido de carbono, compuestos orgánicos volátiles, temperatura y humedad ambiental para la recopilación y transmisión de datos en tiempo real}. El trabajo tendrá un presupuesto preliminar estimado de {600} horas y un costo estimado de {\$ 500}, con fecha de inicio el \fechaInicioName\hspace{1px} y fecha de presentación pública el \fechaFinalName.

Se adjunta a esta acta la planificación inicial.

\vfill

% Esta parte se construye sola con la información que hayan cargado en el preámbulo del documento y no debe modificarla
\begin{table}[ht]
\centering
\begin{tabular}{ccc}
\begin{tabular}[c]{@{}c@{}}Dr. Ing. Ariel Lutenberg \\ Director posgrado FIUBA\end{tabular} & \hspace{2cm} & \begin{tabular}[c]{@{}c@{}}\clientename \\ \empclientename \end{tabular} \vspace{2.5cm} \\ 
\multicolumn{3}{c}{\begin{tabular}[c]{@{}c@{}} \supname \\ Director del Trabajo Final\end{tabular}} \vspace{2.5cm} \\
\end{tabular}
\end{table}




\section{1. Descripción técnica-conceptual del proyecto a realizar}
\label{sec:descripcion}

La contaminación del aire es un problema crítico que afecta tanto a entornos urbanos como industriales, con consecuencias directas sobre la salud pública y el medio ambiente. En Argentina, la situación es alarmante. Según la Organización Mundial de la Salud (OMS), el aire en el país tiene una media anual de 13 µg/m³ de partículas PM2.5, superando en un 30\% el nivel considerado seguro por la organización. En Buenos Aires, esta media anual asciende a 14 µg/m³, lo que implica un 40\% por encima del límite recomendado. Estas cifras se traducen en consecuencias graves, como la muerte anual de 85 niños por enfermedades vinculadas a la contaminación del aire en Argentina.

Este proyecto nace como un emprendimiento personal con el propósito de monitorear la calidad del aire en entornos industriales y urbanos y proporcionar información clave para la toma de decisiones. Actualmente, muchas ciudades y empresas carecen de sistemas eficientes y accesibles para medir en tiempo real la calidad del aire, lo que dificulta la prevención y el control de la contaminación. El objetivo es llenar ese vacío con una solución tecnológica asequible y escalable.

Se propone desarrollar un sistema de monitoreo basado en una red de sensores y la tecnología internet de las cosas (IoT). El sistema será capaz de detectar partículas en el aire tales como PM2.5, niveles de CO2, compuestos químicos dañinos, temperatura y humedad. Los datos obtenidos serán enviados a una plataforma accesible desde un aplicativo web. Los dispositivos estarán conectados mediante LoRaWAN y WiFi/MQTT, almacenarán datos de manera eficiente y se presentarán en una plataforma intuitiva. Revisar la Figura 1 para comprender el diagrama de bloques del sistema de monitoreo.

\begin{figure}[htpb]
\centering 
\includegraphics[width=.65\textwidth]{./Figuras/fig_1-diagBloques.png}
\caption{Diagrama en bloques del sistema.}
\label{fig:diagBloques}
\end{figure}

Este proyecto está diseñado como una solución adaptable, principalmente para implementarse en empresas, con la capacidad de escalar a hogares y gobiernos. Todos los usuarios comparten una preocupación en común, la calidad del aire y su impacto en la salud. El sistema no solo permite el monitoreo en tiempo real, sino también envía alertas cuando los niveles de contaminación superan los límites recomendados. También ofrece herramientas para analizar datos históricos e identificar tendencias, que permitan tomar decisiones oportunas. Más que una simple innovación tecnológica, este proyecto representa una herramienta clave para mejorar la calidad de vida y promover un entorno más saludable y sostenible.

En el mercado actual existen soluciones para el monitoreo de la calidad del aire, sin embargo, muchas de ellas presentan limitaciones, como pueden ser:
\begin{itemize}
	\item Costos elevados: en las etapas de implementación y mantenimiento, un sistema con carácteristicas similares puede volverse solo accesible a instituciones con grandes presupuestos.
	\item Cobertura limitada: debido a que la mayoría de los sensores requieren cableado o dependencias de redes WiFi con alcance reducido.
	\item Falta de integración con plataformas acceesibles: lo que dificulta el análisis y la interpretación de los datos por parte de usuarios sin conocimientos técnicos avanzados.
\end{itemize}

A diferencia de las soluciones, este proyecto estará diseñado como una solución flexible y adaptable a  empresas, hogares y gobiernos. Su propuesta de valor se basa en ofrecer un monitoreo ambiental accesible, para lograr entornos seguros y sostenibles, bajo los siguientes aspectos:

\begin{itemize}
	\item Accesibilidad: una solución asequible en comparación con otros sistemas comerciales.
	\item Escalabilidad: implementación modular, adaptable a distintos entornos y necesidades.
	\item Interfaz intuitiva: plataforma accesible para cualquier usuario, sin necesidad de conocimientos técnicos avanzados.
	\item Conectividad eficiente: uso de tecnologías de comunicación de bajo consumo y gran alcance.
	\item Toma de decisiones informada: alertas y análisis de datos para implementar medidas de mitigación de contaminación.
\end{itemize}

El proyecto se encuentra en una etapa inicial de desarrollo. Para su primera versión se plantea una solución funcional, enfocada en validar su desempeño en entornos reales. En esta fase, el sistema ofrecerá dos modalidades de monitoreo que permitirán a los usuarios gestionar sus dispositivos de manera flexible:  

\begin{itemize}
    \item Monitoreo privado: cada usuario podrá registrar y gestionar sus propios dispositivos, accediendo a la información en tiempo real de los sensores vinculados.  
    \item Monitoreo público: si el usuario así lo decide, podrá compartir los datos recopilados con la comunidad, permitiendo que la información esté disponible en una red abierta. Esto fomentará la creación de un ecosistema colaborativo.  
\end{itemize}

En esta primera versión no se implementarán modelos de suscripción ni esquemas de pago, ya que el objetivo principal es desarrollar un prototipo funcional. Este proyecto permitará evaluar la viabilidad técnica y el impacto del sistema en distintos escenarios de uso.  

A futuro se integrará inteligencia artificial (IA) para análisis predictivo y se aumentarán funcionalidades para mejorar la toma de decisiones urbanas e industriales. El proyecto busca un crecimiento accesible y escalable, garantizando que cada usuario se beneficien de un monitoreo ambiental confiable desde su primera versión.



\vspace{25px}


\section{2. Identificación y análisis de los interesados}
\label{sec:interesados}

\begin{table}[ht]
%caption{Identificación de los interesados}
%\label{tab:interesados}
\begin{tabularx}{\linewidth}{@{}|*{4}{>{\arraybackslash}X|}@{}}
\hline
\rowcolor[HTML]{C0C0C0} 
Rol           & Nombre y Apellido & Organización 	& Puesto 	\\ \hline
% Auspiciante   & -                 & -             	& -       	\\ \hline
Cliente       & \clientename      &\empclientename	&        	\\ \hline
% Impulsor      & -                 & -             	& -       	\\ \hline
Responsable   & \authorname       & FIUBA        	& Alumno 	\\ \hline
% Colaboradores & -                 & -             	& -       	\\ \hline
Orientador    & \supname	      & \pertesupname 	& Director del Trabajo Final \\ \hline
Usuario final & Industrias, hogares o gobiernos	&  Privada o pública & Autoridad o interés en monitoreo ambiental				\\
\hline
% Equipo        & -                 & -             	& -       	\\ \hline
\end{tabularx}
\end{table}

\begin{itemize}
	\item Cliente: Ing. Jose Mauricio Vargas Nuñez es quien propuso los requerimientos del proyecto.
	\item Orientador: \supname es un profesional idóneo para la temática con especialidad en tecnologías IoT y LoRaWAN.
\end{itemize}

\section{3. Propósito del proyecto}
\label{sec:proposito}

El propósito de este proyecto es fomentar entornos más saludables, seguros y sostenibles en zonas urbanas e industriales, a partir de una mejora en la calidad del aire. Esto se logrará mediante un sistema de monitoreo basado en IoT, que permita recopilar, analizar y visualizar datos ambientales en tiempo real. Con esta herramienta, se busca facilitar la identificación de fuentes de contaminación y alertar para que las partes interesadas tomen acciones correctivas oportunas. Además, se busca que la solución sea escalable, accesible y eficiente, lo que asegurao su adaptabilidad a distintos entornos y necesidades.

\section{4. Alcance del proyecto}
\label{sec:alcance}

Este proyecto abarca el desarrollo e implementación de un sistema de monitoreo de calidad del aire basado en IoT. El sistema será capáz de medir, transmitir, almacenar y mostrar datos en tiempo real, proporcionando información clave para la toma de decisiones.

El proyecto incluye:

\begin{itemize}
	\item Sensores para medir partículas en suspensión, dióxido de carbono, compuestos orgánicos volátiles, temperatura y humedad.
	\item Microcontrolares ESP32 con modulo LoRa y antena para comunicación se con el servidor mediante protocolos de LoRaWAN, WiFi, MQTT y TCP/IP.
	\item Servidor AWS IoT Core, usando brocker Mosquito y el diseño de la base de datos SQL/NoSQL con PostgreSQL.
	\item El diseño de un aplicativo web para las visualizaciones del usuario, donde se veran los dispositivos conectados, mediciones, gráficas de históricos y alertas.
	
	
\end{itemize}

El presente proyecto no incluye:
\begin{itemize}
	\item Desarrollo de modelos de suscripción o monetización, ya que en esta fase el enfoque es la validación del prototipo.
	\item Integración con inteligencia artificial o modelos predictivos avanzados.
	\item Implementación de una red de sensores a gran escala más allá del piloto inicial.
	\item Certificaciones oficiales de calidad del aire, ya que el sistema servirá como referencia complementaria a mediciones gubernamentales o institucionales.
	\item Localización geográfica que marque el estado de las zonas monitoreadas.
\end{itemize}

El alcance del proyecto está limitado a ser considerado un prototipo, enfocado en validar la viabilidad técnica y operativa. Las futuras versiones podrán incorporar mejoras basadas en los resultados de esta etapa.

\section{5. Supuestos del proyecto}
\label{sec:supuestos}

Para el desarrollo del presente proyecto se supone que:

\begin{itemize}
	\item Disponibilidad financiera: dado que se trata de un emprendimiento personal, los costos del proyecto podrán ser cubiertos por el responsable del proyecto.
	\item Disponibilidad tecnológica: se contará con acceso a los sensores, microcontroladores y todos los componentes necesarios para la fabricación de los dispositovos IoT.
	\item Tiempo: El responsable cumplirá con la planificación propuesta, evitará retrasos y culminará el proyecto de manera satisfactoria.
	\item Infraestructura de comunicación: Podrán realizarse pruebas en entornos controlados pertinentes para el correcto desarrollo del proyecto.
	\item Servidores cloud: Se contará con espacios de prueba en AWS, los dominios y servicios para utilizar el brocker Mosquito.

\end{itemize}

\section{6. Requerimientos}
\label{sec:requerimientos}

\begin{enumerate}

	\item Requerimientos funcionales:
	\begin{enumerate}
	

		\item El sistema permitirá la captura periódica de datos ambientales a través de sensores conectados a dispositivos IoT.
		\item Los datos se transmitirán al servidor en tiempo real utilizando LoRaWAN o WiFi/MQTT según el entorno.
		\item El sistema deberá almacenar los datos para su posterior análisis.
		\item Se deberá ofrecer la opción de publicar los datos de manera privada o pública.
		\item La interfaz será capaz de mostrar a los usuarios las mediciones de sus dispositivos y las zonas que tenga registrado un usuario.
		\item Deberá generarse una alerta cuando los valores superen los umbrales definidos como peligrosos.
	\end{enumerate}
	
	\item Requerimientos de infraestructura
	\begin{enumerate}
		\item Se deberá contar con acceso a una red WiFi estable o cobertura LoRaWAN en las zonas de instalación de los dispositivos.
		\item El sistema requerirá un servidor (local o en la nube) con capacidad para recibir, procesar y almacenar datos provenientes de múltiples nodos.
		\item La base de datos utilizada deberá estar optimizada para el manejo de series temporales y para la gestión estructurada de usuarios y configuraciones.
		\item La alimentación de los dispositivos deberá realizarse mediante batería recargable o fuente USB, con posibilidad de alimentación solar para aplicaciones en exteriores.	
	\end{enumerate}
	
	\item Requerimientos de documentación
	\begin{enumerate}
		\item Deberá documentarse el proceso de montaje del hardware, incluyendo la conexión de sensores al microcontrolador ESP32-S3 y la configuración de red.
		\item Se deberá incluir un manual de instalación y uso de la plataforma web.
		\item La documentación deberá contemplar las APIs utilizadas o desarrolladas para la comunicación entre dispositivos y servidor.	
	\end{enumerate}

	\item Requerimientos del entregable
	\begin{enumerate}
		\item El entregable consistirá en un prototipo funcional que incluya al menos dos dispositivos que realicen las mediciones ambientales, un nodo IoT operativo, una 		\item plataforma web de visualización, un sistema de almacenamiento de datos y un módulo de alertas.
		\item El sistema deberá estar probado en un entorno controlado y documentado como prueba de concepto.
		\item Se desarrollará la memoria final del proyecto.

	\end{enumerate}
	
	\item Requerimientos de la interfaz
	\begin{enumerate}
		\item La plataforma deberá contar con una interfaz web responsiva para el monitoreo.
		\item La interfaz permitirá visualizar datos en tiempo real, acceder a históricos, gestionar dispositivos y configurar alertas.
		\item Los indicadores de calidad del aire deberán representarse mediante códigos de colores intuitivos y comprensibles.
		\item El sistema ofrecerá una interfaz de visualización con lecturas de datos ambientales de las zonas definidas como públicas. Se limitará la visualización al estado de las zonas sin tener acceso a los dispositivos vinculados y datos históricos.

	\end{enumerate}

	\item Requerimientos funcionales del sistema para el rol “Usuario”
	\begin{enumerate}
		\item El usuario deberá poder registrarse y asociar dispositivos a su cuenta personal.
		\item El usuario gestionará sus dispositivos pudiendo agregar, eliminar, editar y asignar la ubicación en zonas definidas.
		\item Tendrá acceso a la visualización en tiempo real de los datos capturados por sus sensores.
		\item Podrá decidir si desea compartir sus datos de forma pública o mantenerlos en modo privado.
		\item Recibirá alertas personalizadas cuando los valores medidos superen los umbrales definidos.

	\end{enumerate}

	\item Requerimientos funcionales del sistema para el rol “Administrador”
	\begin{enumerate}
		\item El administrador deberá tener acceso a la gestión de usuarios, dispositivos y configuraciones generales del sistema.
		\item Tendrá visibilidad completa sobre las métricas generadas por todos los nodos activos.
		\item Deberá poder acceder a registros (logs) del sistema y supervisar el estado de funcionamiento de cada dispositivo.

	\end{enumerate}

	\item Requerimientos funcionales del sistema para la vista pública
	\begin{enumerate}
		\item La persona que quiera tener acceso a la vista pública deberá llenar un formulario donde se exprese su intención de acceder a esta información.
		\item La interfaz pública tendrá a disposición el estado en tiempo real de las mediciones ambientales de las zonas públicas.

	\end{enumerate}

	\item Requerimientos de seguridad
	\begin{enumerate}
		\item El sistema gestionará las credenciales para el ingreso de los usuarios.
		\item Se contará con métodos para recuperación de contraseñas y verificación de usuarios.
		\item La información asociada a los usuarios y a dispositivos configurados como privados deberá mantenerse protegida y no podrá hacerse pública sin consentimiento expreso.

	\end{enumerate}

\end{enumerate}

\section{7. Historias de usuarios (\textit{Product backlog})}
\label{sec:backlog}

\begin{enumerate}
\item
Como usuario del sistema, quiero registrar un dispositivo IoT en mi cuenta para poder acceder a sus datos ambientales.

Complejidad: 3
Dificultad: 2
Incertidumbre: 1
Suma: 6 → Story Points: 8

Criterios de aceptación:
\begin{itemize}
	\item El dispositivo queda vinculado a la cuenta del usuario al ingresar su ID único.
	\item El dispositivo aparece en el panel de la cuenta una vez registrado.
	\item El backend guarda la asociación en la base de datos y la valida en cada consulta.
\end{itemize}

\item
Como usuario del sistema, quiero poder ver todos los dispositivos registrados para monitorear su estado general.

Complejidad: 3
Dificultad: 3
Incertidumbre: 3
Suma: 9 → Story Points: 13

Criterios de aceptación:
\begin{itemize}
	\item Se lista el total de dispositivos activos/inactivos.
	\item El estado de conexión de cada dispositivo es visible mediante un ícono claro.
	\item Los dispositivos se consultan mediante una API segura y autenticada.
\end{itemize}


\item Como usuario, quiero visualizar los datos en tiempo real de mis sensores para conocer el estado del ambiente.

Complejidad: 3
Dificultad: 3
Incertidumbre: 3
Suma: 9 → Story Points: 13

Criterios de aceptación:
\begin{itemize}
	\item Los datos se actualizan en intervalos definidos.
	\item Los valores se presentan en tarjetas con colores según el nivel de calidad del aire.
	\item La plataforma consulta la API de datos en tiempo real.
\end{itemize}

\item Como usuario, quiero acceder al historial de datos recolectados para analizar la evolución de la calidad del aire.

Complejidad: 4
Dificultad: 3
Incertidumbre: 3
Suma: 10 → Story Points: 13

Criterios de aceptación:
\begin{itemize}
	\item Se puede seleccionar un rango de fechas y consultar datos históricos.
	\item Los datos se grafican con líneas de tendencia y filtros por parámetro.
	\item La base de datos devuelve los datos en bloques optimizados para series temporales.
\end{itemize}

\item Como usuario, quiero configurar umbrales para cada sensor para recibir alertas ante niveles peligrosos.

Complejidad: 3
Dificultad: 2
Incertidumbre: 3
Suma: 8 → Story Points: 8

Criterios de aceptación:

\begin{itemize}
	\item Se pueden definir umbrales distintos para cada parámetro.
	\item El usuario recibe un mensaje claro en la interfaz si se supera un umbral.
	\item Las reglas se almacenan por usuario y se validan en tiempo real.
\end{itemize}

\item Como usuario, quiero recibir una notificación en pantalla si se detecta contaminación peligrosa.

Complejidad: 2
Dificultad: 2
Incertidumbre: 1
Suma: 5 → Story Points: 5

Criterios de aceptación:

\begin{itemize}
	\item El sistema envía automáticamente una notificación si se supera un umbral.
	\item El mensaje contiene el parámetro, el valor registrado y la hora.
	\item El envío se realiza mediante un alert.

\end{itemize}

\item Como usuario, quiero poder elegir si mis datos son públicos o privados para decidir con quién los comparto.

Complejidad: 2
Dificultad: 2
Incertidumbre: 1
Suma: 5 → Story Points: 5

Criterios de aceptación:
\begin{itemize}
	\item El usuario puede alternar entre visibilidad pública y privada desde la configuración del dispositivo.
	\item Un ícono o leyenda indica claramente el estado actual.
	\item La API restringe el acceso a los datos privados a usuarios autenticados.

\end{itemize}

\item Como visitante del sitio web, quiero poder visualizar los datos públicos de calidad del aire organizados por zonas, para conocer el estado ambiental.

Complejidad: 3
Dificultad: 3
Incertidumbre: 3
Suma: 9 → Story Points: 13

Criterios de aceptación:

\begin{itemize}
	\item El sistema permite acceso libre a una sección pública donde se muestran los datos de la calidad del aire correspondientes a una zona con visibilidad pública.
	\item Los datos se organizan por zona, haciendo referencia de su dirección y se muestra la última lectura de cada medición ambiental.
	\item No requiere autenticación y los datos se actualizan con una frecuencia configurable.

\end{itemize}


\end{enumerate}


\section{8. Entregables principales del proyecto}
\label{sec:entregables}

\begin{itemize}
\item Aplicativo web de visualización

\item Código fuente

\item Base de datos operativa

\item Manual de usuario

\item Manual de instalación

\item Esquemático de conexión de sensores

\item Diagrama de bloques del sistema

\item Memoria final del trabajo

\end{itemize}

\section{9. Desglose del trabajo en tareas}
\label{sec:wbs}

\begin{consigna}{red}
El WBS debe tener relación directa o indirecta con los requerimientos.  Son todas las actividades que se harán en el proyecto para dar cumplimiento a los requerimientos. Se recomienda mostrar el WBS mediante una lista indexada:

\begin{enumerate}
\item Grupo de tareas 1 (suma h)
	\begin{enumerate}
	\item Tarea 1 (tantas h)
	\item Tarea 2 (tantas h)
	\item Tarea 3 (tantas h)
	\end{enumerate}
\item Grupo de tareas 2 (suma h)
	\begin{enumerate}
	\item Tarea 1 (tantas h)
	\item Tarea 2 (tantas h)
	\item Tarea 3 (tantas h)
	\end{enumerate}
\item Grupo de tareas 3 (suma h)
	\begin{enumerate}
	\item Tarea 1 (tantas h)
	\item Tarea 2 (tantas h)
	\item Tarea 3 (tantas h)
	\item Tarea 4 (tantas h)
	\item Tarea 5 (tantas h)
	\end{enumerate}
\end{enumerate}

Cantidad total de horas: tantas.

\textbf{¡Importante!:} la unidad de horas es h y va separada por espacio del número. Es incorrecto escribir ``23hs".

\textbf{Se recomienda que no haya ninguna tarea que lleve más de 40 h.} De ser así se recomienda dividirla en tareas de menor duración.

\end{consigna}

\section{10. Diagrama de Activity On Node}
\label{sec:AoN}

\begin{consigna}{red}
Armar el AoN a partir del WBS definido en la etapa anterior.

Una herramienta simple para desarrollar los diagramas es el Draw.io (\url{https://app.diagrams.net/}).
\href{https://app.diagrams.net}{Draw.io}


\begin{figure}[htpb]
\centering 
\includegraphics[width=.8\textwidth]{./Figuras/AoN.png}
\caption{Diagrama de \textit{Activity on Node}.}
\label{fig:AoN}
\end{figure}

Indicar claramente en qué unidades están expresados los tiempos.
De ser necesario indicar los caminos semi críticos y analizar sus tiempos mediante un cuadro.
Es recomendable usar colores y un cuadro indicativo describiendo qué representa cada color.

\end{consigna}

\section{11. Diagrama de Gantt}
\label{sec:gantt}

\begin{consigna}{red}
Existen muchos programas y recursos \textit{online} para hacer diagramas de Gantt, entre los cuales destacamos:

\begin{itemize}
\item Planner
\item GanttProject
\item Trello + \textit{plugins}. En el siguiente link hay un tutorial oficial: \\ \url{https://blog.trello.com/es/diagrama-de-gantt-de-un-proyecto}
\item Creately, herramienta online colaborativa. \\\url{https://creately.com/diagram/example/ieb3p3ml/LaTeX}
\item Se puede hacer en latex con el paquete \textit{pgfgantt}\\ \url{http://ctan.dcc.uchile.cl/graphics/pgf/contrib/pgfgantt/pgfgantt.pdf}
\end{itemize}

Pegar acá una captura de pantalla del diagrama de Gantt, cuidando que la letra sea suficientemente grande como para ser legible. 
Si el diagrama queda demasiado ancho, se puede pegar primero la ``tabla'' del Gantt y luego pegar la parte del diagrama de barras del diagrama de Gantt.

Configurar el software para que en la parte de la tabla muestre los códigos del EDT (WBS).\\
Configurar el software para que al lado de cada barra muestre el nombre de cada tarea.\\
Revisar que la fecha de finalización coincida con lo indicado en el Acta Constitutiva.

En la figura \ref{fig:gantt}, se muestra un ejemplo de diagrama de gantt realizado con el paquete de \textit{pgfgantt}. 
En la plantilla pueden ver el código que lo genera y usarlo de base para construir el propio.

Las fechas pueden ser calculadas utilizando alguna de las herramientas antes citadas. Sin embargo, el siguiente ejemplo
fue elaborado utilizando 
\href{https://docs.google.com/spreadsheets/d/1fBz8NhSpc4tkkhz3KjJCbh1nR_ltDkfEcZi4tZXduqs}{esta hoja de cálculo}.

Es importante destacar que el ancho del diagrama estará dado por la longitud del texto utilizado para las tareas 
(Ejemplo: tarea 1, tarea 2, etcétera) y el valor \textit{x unit}. Para mejorar la apariencia del diagrama, es necesario
ajustar este valor y, quizás, acortar los nombres de las tareas.

\begin{figure}[htpb]
  \begin{center}
    \begin{ganttchart}[
      time slot unit=day,
      time slot format=isodate,
      x unit=0.038cm,
      y unit title=0.7cm,
      y unit chart=0.6cm,
      milestone/.append style={xscale=4}
      ]{2021-03-05}{2021-12-16}
      \gantttitlecalendar*{2021-03-05}{2021-12-16}{year} \\
      \gantttitlecalendar*{2021-03-05}{2021-12-16}{month} \\
      \ganttgroup{Duración Total}{2021-03-05}{2021-12-16} \\
      %%%%%%%%%%%%%%%%%Organización
      \ganttgroup{Organización}{2021-03-05}{2021-04-16} \\
      \ganttbar{Planificación del proyecto}{2021-03-05}{2021-04-15} \\
      %%%%%%%%%%%%%%%%%Ejecución
      \ganttgroup{Ejecución}{2021-04-16}{2021-10-21} \\
      \ganttbar{Tarea 1}{2021-04-16}{2021-04-29} \\
      \ganttbar{Tarea 2}{2021-04-30}{2021-05-13} \\
      \ganttbar{Tarea 3}{2021-05-14}{2021-05-27} \\
      \ganttbar{Tarea 4}{2021-05-28}{2021-07-12} \\
      \ganttbar{Tarea 5}{2021-07-13}{2021-08-09} \\
      \ganttbar{Tarea 6}{2021-08-10}{2021-09-23} \\
      \ganttbar{Tarea 7}{2021-09-24}{2021-09-30} \\
      \ganttbar{Tarea 8}{2021-10-01}{2021-10-14} \\
      \ganttbar{Tarea 9}{2021-10-15}{2021-10-21} \\
      % %%%%%%%%%%%%%%%%%Finalización
      \ganttgroup{Finalización}{2021-10-22}{2021-12-16} \\
      \ganttbar{Memoria v1}{2021-10-22}{2021-11-04} \\
      \ganttbar{Memoria v2}{2021-11-05}{2021-11-18} \\
      \ganttbar{Memoria final}{2021-11-19}{2021-12-02} \\
      % La fecha del siguiente milestone es la fecha en que terminamos la memoria
      \ganttmilestone{Enviar memoria al director}{2021-12-02} \\
      \ganttbar{Elaborar la presentación}{2021-12-03}{2021-12-16} \\
      \ganttmilestone{Ensayo de la presentación}{2021-12-16} \\
      %%%%%%%%%%%%%%%%%%%%%%%%%%%%%%%%%%%%%%%%%%%%%%%%%%%%%%%%%%%%%%%
    \end{ganttchart}
  \end{center}
  \caption{Diagrama de gantt de ejemplo}
  \label{fig:gantt}
\end{figure}


\begin{landscape}
\begin{figure}[htpb]
\centering 
\includegraphics[height=.85\textheight]{./Figuras/Gantt-2.png}
\caption{Ejemplo de diagrama de Gantt (apaisado).} %Modificar este título acorde.
\label{fig:diagGantt}
\end{figure}

\end{landscape}

\end{consigna}


\section{12. Presupuesto detallado del proyecto}
\label{sec:presupuesto}

\begin{consigna}{red}
Si el proyecto es complejo entonces separarlo en partes:
\begin{itemize}
	\item Un total global, indicando el subtotal acumulado por cada una de las áreas.
	\item El desglose detallado del subtotal de cada una de las áreas.
\end{itemize}

IMPORTANTE: No olvidarse de considerar los COSTOS INDIRECTOS.

Incluir la aclaración de si se emplea como moneda el peso argentino (ARS) o si se usa moneda extranjera (USD, EUR, etc). Si es en moneda extranjera se debe indicar la tasa de conversión respecto a la moneda local en una fecha dada.

\end{consigna}

\begin{table}[htpb]
\centering
\begin{tabularx}{\linewidth}{@{}|X|c|r|r|@{}}
\hline
\rowcolor[HTML]{C0C0C0} 
\multicolumn{4}{|c|}{\cellcolor[HTML]{C0C0C0}COSTOS DIRECTOS} \\ \hline
\rowcolor[HTML]{C0C0C0} 
Descripción &
  \multicolumn{1}{c|}{\cellcolor[HTML]{C0C0C0}Cantidad} &
  \multicolumn{1}{c|}{\cellcolor[HTML]{C0C0C0}Valor unitario} &
  \multicolumn{1}{c|}{\cellcolor[HTML]{C0C0C0}Valor total} \\ \hline
 &
  \multicolumn{1}{c|}{} &
  \multicolumn{1}{c|}{} &
  \multicolumn{1}{c|}{} \\ \hline
 &
  \multicolumn{1}{c|}{} &
  \multicolumn{1}{c|}{} &
  \multicolumn{1}{c|}{} \\ \hline
\multicolumn{1}{|l|}{} &
   &
   &
   \\ \hline
\multicolumn{1}{|l|}{} &
   &
   &
   \\ \hline
\multicolumn{3}{|c|}{SUBTOTAL} &
  \multicolumn{1}{c|}{} \\ \hline
\rowcolor[HTML]{C0C0C0} 
\multicolumn{4}{|c|}{\cellcolor[HTML]{C0C0C0}COSTOS INDIRECTOS} \\ \hline
\rowcolor[HTML]{C0C0C0} 
Descripción &
  \multicolumn{1}{c|}{\cellcolor[HTML]{C0C0C0}Cantidad} &
  \multicolumn{1}{c|}{\cellcolor[HTML]{C0C0C0}Valor unitario} &
  \multicolumn{1}{c|}{\cellcolor[HTML]{C0C0C0}Valor total} \\ \hline
\multicolumn{1}{|l|}{} &
   &
   &
   \\ \hline
\multicolumn{1}{|l|}{} &
   &
   &
   \\ \hline
\multicolumn{1}{|l|}{} &
   &
   &
   \\ \hline
\multicolumn{3}{|c|}{SUBTOTAL} &
  \multicolumn{1}{c|}{} \\ \hline
\rowcolor[HTML]{C0C0C0}
\multicolumn{3}{|c|}{TOTAL} &
   \\ \hline
\end{tabularx}%
\end{table}


\section{13. Gestión de riesgos}
\label{sec:riesgos}

\begin{consigna}{red}
a) Identificación de los riesgos (al menos cinco) y estimación de sus consecuencias:
 
Riesgo 1: detallar el riesgo (riesgo es algo que si ocurre altera los planes previstos de forma negativa)
\begin{itemize}
	\item Severidad (S): mientras más severo, más alto es el número (usar números del 1 al 10).\\
	Justificar el motivo por el cual se asigna determinado número de severidad (S).
	\item Probabilidad de ocurrencia (O): mientras más probable, más alto es el número (usar del 1 al 10).\\
	Justificar el motivo por el cual se asigna determinado número de (O). 
\end{itemize}   

Riesgo 2:
\begin{itemize}
	\item Severidad (S): X.\\
	Justificación...
	\item Ocurrencia (O): Y.\\
	Justificación...
\end{itemize}

Riesgo 3:
\begin{itemize}
	\item Severidad (S):  X.\\
	Justificación...
	\item Ocurrencia (O): Y.\\
	Justificación...
\end{itemize}


b) Tabla de gestión de riesgos:      (El RPN se calcula como RPN=SxO)

\begin{table}[htpb]
\centering
\begin{tabularx}{\linewidth}{@{}|X|c|c|c|c|c|c|@{}}
\hline
\rowcolor[HTML]{C0C0C0} 
Riesgo & S & O & RPN & S* & O* & RPN* \\ \hline
       &   &   &     &    &    &      \\ \hline
       &   &   &     &    &    &      \\ \hline
       &   &   &     &    &    &      \\ \hline
       &   &   &     &    &    &      \\ \hline
       &   &   &     &    &    &      \\ \hline
\end{tabularx}%
\end{table}

Criterio adoptado: 

Se tomarán medidas de mitigación en los riesgos cuyos números de RPN sean mayores a...

Nota: los valores marcados con (*) en la tabla corresponden luego de haber aplicado la mitigación.

c) Plan de mitigación de los riesgos que originalmente excedían el RPN máximo establecido:
 
Riesgo 1: plan de mitigación (si por el RPN fuera necesario elaborar un plan de mitigación).
  Nueva asignación de S y O, con su respectiva justificación:
  \begin{itemize}
	\item Severidad (S*): mientras más severo, más alto es el número (usar números del 1 al 10).
          Justificar el motivo por el cual se asigna determinado número de severidad (S).
	\item Probabilidad de ocurrencia (O*): mientras más probable, más alto es el número (usar del 1 al 10).
          Justificar el motivo por el cual se asigna determinado número de (O).
	\end{itemize}

Riesgo 2: plan de mitigación (si por el RPN fuera necesario elaborar un plan de mitigación).
 
Riesgo 3: plan de mitigación (si por el RPN fuera necesario elaborar un plan de mitigación).

\end{consigna}


\section{14. Gestión de la calidad}
\label{sec:calidad}

\begin{consigna}{red}
Elija al menos diez requerimientos que a su criterio sean los más importantes/críticos/que aportan más valor y para cada uno de ellos indique las acciones de verificación y validación que permitan asegurar su cumplimiento.

\begin{itemize} 
\item Req \#1: copiar acá el requerimiento con su correspondiente número.

\begin{itemize}
	\item Verificación para confirmar si se cumplió con lo requerido antes de mostrar el sistema al cliente. Detallar.
	\item Validación con el cliente para confirmar que está de acuerdo en que se cumplió con lo requerido. Detallar. 
\end{itemize}

\end{itemize}

Tener en cuenta que en este contexto se pueden mencionar simulaciones, cálculos, revisión de hojas de datos, consulta con expertos, mediciones, etc.  

Las acciones de verificación suelen considerar al entregable como ``caja blanca'', es decir se conoce en profundidad su funcionamiento interno.  

En cambio, las acciones de validación suelen considerar al entregable como ``caja negra'', es decir, que no se conocen los detalles de su funcionamiento interno.

\end{consigna}

\section{15. Procesos de cierre}    
\label{sec:cierre}

\begin{consigna}{red}
Establecer las pautas de trabajo para realizar una reunión final de evaluación del proyecto, tal que contemple las siguientes actividades:

\begin{itemize}
	\item Pautas de trabajo que se seguirán para analizar si se respetó el Plan de Proyecto original:\\
	 - Indicar quién se ocupará de hacer esto y cuál será el procedimiento a aplicar. 
	\item Identificación de las técnicas y procedimientos útiles e inútiles que se emplearon, los problemas que surgieron y cómo se solucionaron:\\
	 - Indicar quién se ocupará de hacer esto y cuál será el procedimiento para dejar registro.
	\item Indicar quién organizará el acto de agradecimiento a todos los interesados, y en especial al equipo de trabajo y colaboradores:\\
	  - Indicar esto y quién financiará los gastos correspondientes.
\end{itemize}

\end{consigna}

\end{document}